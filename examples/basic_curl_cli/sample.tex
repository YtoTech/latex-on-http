% !TeX program = pdflatex
\documentclass[varwidth=210mm, border=0pt]{standalone}
\usepackage[utf8]{inputenc}
\usepackage[T2A]{fontenc}
\usepackage[russian]{babel}
\usepackage{tikz}
\usepackage{xcolor}
\usepackage{pagecolor}
\usetikzlibrary{decorations.text, shapes.geometric, backgrounds}

% Цвета
\definecolor{bgred}{HTML}{B71C1C}
\definecolor{badge}{HTML}{7A0F0F}
\definecolor{wood}{HTML}{8B5A2B}
\definecolor{wood2}{HTML}{D0A06A}
\definecolor{metal}{HTML}{2B2B2B}
\definecolor{fire1}{HTML}{FF6F00}
\definecolor{fire2}{HTML}{FFD54F}
\definecolor{spike}{HTML}{111111}

\pagecolor{bgred}
\begin{document}
\begin{tikzpicture}[scale=2, every node/.style={transform shape}]
  % Центр и размеры
  \coordinate (C) at (0,0);
  \def\R{2} % радиус круглого бэйджа

  % Круглый бэйдж (слегка тон-у тон)
  \shade[ball color=badge!90!black] (C) circle (\R);
  \fill[white, opacity=0.02] (C) circle (\R-0.02);

  % Внутренний круг для контраста
  \fill[black!8, draw=black!30, line width=0.6pt] (C) circle (\R-0.3);

  % Крест-накрест: два деревянных молотка
  % Параметры молотка
  \def\handlelen{2.2}
  \def\handlewid{0.12}
  \def\headlen{0.7}
  \def\headwid{0.45}
  \def\spikecount{5}

  % Функция: рисуем один молоток, ориентированный по углу #1 градусов, сдвиг по центру
  \newcommand{\drawmallet}[2]{% #1 = angle, #2 = scale sign (1 or -1) for minor offset
    \begin{scope}[rotate=#1]
      % Ручка
      \shade[bottom color=wood!60, top color=wood2] ( -\handlelen, -\handlewid) rectangle (0, \handlewid);
      \draw[line width=0.6pt, brown!60] ( -\handlelen, -\handlewid) rectangle (0, \handlewid);
      % Скругление хвоста ручки
      \fill[wood] (-\handlelen-0.08,-\handlewid) ++(0,0) circle (0.08);
      % Головка молотка
      \begin{scope}[shift={( -0.1,0)}, xshift=0pt]
        \fill[metal!15!black] (0.05, -\headwid) rectangle (\headlen, \headwid);
        \draw[line width=0.8pt] (0.05, -\headwid) rectangle (\headlen, \headwid);
        % Шипы сверху
        \foreach \i in {0,...,4} {
          \pgfmathsetmacro{\x}{0.05 + (\i+0.3)*(\headlen-0.05)/\spikecount}
          \path[fill=spike] (\x, \headwid) -- (\x+0.06, \headwid+0.18) -- (\x+0.12, \headwid) -- cycle;
        }
        % Шипы снизу
        \foreach \i in {0,...,4} {
          \pgfmathsetmacro{\x}{0.05 + (\i+0.3)*(\headlen-0.05)/\spikecount}
          \path[fill=spike] (\x, -\headwid) -- (\x+0.06, -\headwid-0.18) -- (\x+0.12, -\headwid) -- cycle;
        }
      \end{scope}

      % Небольшие трещины и текстура на ручке
      \draw[line width=0.4pt, brown!40] (-0.4,0.05) to[out=10,in=190] (-1.0,0.06);
      \draw[line width=0.4pt, brown!40] (-0.9,-0.03) to[out=-10,in=210] (-1.6,-0.04);
    \end{scope}
  }

  % Рисуем два молотка крест-накрест
  \begin{scope}
    \drawmallet{30}{1}
    \drawmallet{-35}{-1}
  \end{scope}

  % Огонь, исходящий из-за молотков
  \begin{scope}
    \clip (C) circle (\R-0.35);
    % Слой огня — несколько форм
    \fill[fire2] (-1.1,-0.15) .. controls (-0.9,0.8) and (-0.4,1.2) .. (0,1.05) .. controls (0.35,0.95) and (0.7,0.6) .. (0.9,0.15) -- (0.9,-0.6) .. controls (0.55,-0.2) and (0.2,-0.4) .. (-0.1,-0.25) -- cycle;
    \fill[fire1] (-0.95,-0.05) .. controls (-0.75,0.6) and (-0.35,0.9) .. (0,0.85) .. controls (0.3,0.78) and (0.6,0.5) .. (0.75,0.18) -- (0.75,-0.4) .. controls (0.45,-0.05) and (0.15,-0.25) .. (-0.05,-0.15) -- cycle;
    % Внутренние языки
    \fill[yellow!80!orange] (-0.5,0.03) .. controls (-0.3,0.45) and (-0.12,0.6) .. (0,0.55) .. controls (0.09,0.52) and (0.22,0.4) .. (0.28,0.3) -- (0.28,-0.05) .. controls (0.05,0.1) and (-0.15,0.0) .. (-0.3,0.01) -- cycle;
  \end{scope}

  % Текст по дуге внизу (название)
  \begin{scope}
    \def\textstr{КОЛОТУШКИ}
    \draw[white] (0,-1.2) arc (200:340:1.4) node[midway, sloped, below, text=white, font=\bfseries\Large]{\textstr};
    % Более аккуратно — с использованием decorations.text
    \path[decorate, decoration={text along path, text={\bfseries\Large \textstr}, raise=3pt}] (180:-1.45) arc (180:0:1.45);
  \end{scope}

  % Маленькая центральная эмблема (опционально)
  \fill[black!10] (0,0.9) circle (0.18);
  \draw[line width=0.6pt, white] (0,0.9) circle (0.18);
  \node[text=white, font=\bfseries\small] at (0,0.9) {KS};

  % Контур внешнего круга
  \draw[line width=1.2pt, color=black!50] (C) circle (\R);

\end{tikzpicture}
\end{document}

